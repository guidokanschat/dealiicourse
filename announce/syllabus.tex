\documentclass{article}
\usepackage{hyperref}
\parindent0pt
\parskip 1ex plus 1ex
\addtolength{\topmargin}{-1in}
\addtolength{\textheight}{1.5in}
\begin{document}
\title{Partial differential equations with deal.II.\\A short course}
\author{Guido Kanschat, IWR, Heidelberg University}
\date{Pacific Institute for Mathematical Sciences (PIMS)\\August 2016}
\maketitle
\begin{abstract}
  In this class, we introduce into using the finite element library
  deal.II in order to solve partial differential equations. The class
  covers basic topics from installing the library and adding support
  for auxiliary software to setting up a mesh. It advances to defining
  various finite element spaces on such a mesh and the implementation
  of bilinear forms. We will cover the implementation of discontinuous
  Galerkin methods as well as multigrid solvers and
  preconditioners. The capabilities of deal.II for multithreading and
  message passing paralelization will be introduced.  The course
  discusses applications like potential problems, linear and nonlinear
  elasticity, incompressible flow, porous media flow, Maxwell
  eigenvalue problems, and possibly applications contributed by the
  audience.
\end{abstract}

\subsection*{Course outline}

The course will be given in 10 lectures with time for programming
after each lecture. Participants should have deal.II and amandus installed 

\begin{enumerate}
\item Grand overview of deal.II and details of the installation process.
\item Creating and refining meshes. Setting up finite element spaces.
\item A first Poisson solver
\item A state-of-the-art multilevel Poisson solver with discontinuous
  Galerkin methods
\item I have an equation and want to solve it. No more. Amandus.
\item Mixed finite elements
\item Eigenvalue problems
\item Exploring error estimation and adaptive refinement
\item Multigrid preconditioning
\item Parallelization
\end{enumerate}
Deviations from this schedule can be agreed on with the participants.

\subsection*{Prerequisites}
Participants should have a basic general knowledge of the finite
element method as well as of C++.

Software should be downloaded (see below) and tested on participants'
computers.


\subsection*{Downloading and installation}

We will use the program package deal.II and the application framework
amandus which is built on top of it. It will simplify and speed up the
progress of participants considerably, if both are already
preinstalled on the participants' computers.

The widely used (more than 500 downloads per month) finite element
library deal.II can be downloaded from
\begin{center}
 \url{www.dealii.org}. 
\end{center}
For this class, we recommend following the instructions for the
developer version, which involves cloning from github. For beginners,
it is recommended to install deal.II as standard installation without
support for MPI, Petsc and Trilinos. If parallelization is an
immediate objective, MPI and Trilinos support should be
activated. Petsc is only recommended if it is preinstalled on your
computer.

Amandus is a small package simplifying the implementation of a lot of
applications. It can be cloned from bitbucket at
\begin{center}
  \url{https://bitbucket.org/guidokanschat/amandus}.
\end{center}
Instructions for installation AFTER deal.II can be found under the
link above.

\end{document}

%%% Local Variables:
%%% mode: latex
%%% TeX-master: t
%%% End:
