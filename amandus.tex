\section[Amandus]{The Amandus project}
\frame{\tableofcontents[currentsection,hideothersubsections]}
\subsection[Installation]{Download and installation}
\begin{frame}
  \frametitle{Downloading from bitbucket.org}
  From your home directory, run the following:
  \begin{block}{}
    \lstinputlisting[basicstyle=\footnotesize]{code/vmamandus.sh}
  \end{block}
\end{frame}

\begin{frame}
  \frametitle{Goals}
  \begin{itemize}
  \item Avoid reimplementing and copying code
    \begin{itemize}
    \item Profit from updates faster
    \end{itemize}
  \item Generic and simple to use interface for
    \begin{itemize}
    \item Newton iterations
    \item Timestepping
    \item Eigenvalues
    \item Multigrid
    \end{itemize}
  \item Computing interesting things with fast learning curve
  \end{itemize}
\end{frame}

\subsection[Library]{The library classes and functions}
\frame{\tableofcontents[currentsection,subsectionstyle=show/shaded/hide]}

\begin{frame}
  \frametitle{The Application classes}
  Encapsulate in main classes
  \begin{columns}
    \begin{column}{.5\textwidth}
      \begin{itemize}
      \item mesh and \lstinline!DoFHandler!
      \item constraints for hanging nodes and boundary
      \item matrices and vectors
      \item linear solver (GMRES)
      \item quadrature
      \item parameters for control at runtime
      \end{itemize}
    \end{column}
    \begin{column}{.5\textwidth}
      \begin{itemize}
      \item Assembling of matrices and residuals
        \begin{itemize}
        \item CG, DG, scalar and vector valued
        \item automatic choice of quadrature etc.
        \end{itemize}
      \item Error estimation and evaluation
      \item Multigrid or UMFPack solvers
      \end{itemize}
    \end{column}
  \end{columns}
\end{frame}

\begin{frame}
  \frametitle{The loops}
  \begin{itemize}
  \item Loop over a sequence of meshes
    \begin{itemize}
    \item global refinement
    \item adaptive refinement (requires estimator)
    \end{itemize}
  \item Different solvers require different loops
    \begin{itemize}
    \item linear
    \item nonlinear
    \item eigenvalue
    \end{itemize}
  \end{itemize}
\end{frame}

\subsection[Models]{The model directories}
\frame{\tableofcontents[currentsection,subsectionstyle=show/shaded/hide]}

\begin{frame}
  \frametitle{Subdirectories for equations}
  \begin{itemize}
  \item \lstinline!laplace!
  \item \lstinline!stokes!, \lstinline!elasticity!
  \item \lstinline!maxwell!
  \item \lstinline!advection!
  \item \lstinline!darcy!, \lstinline!brinkman!
  \item \lstinline!lotkavolterra!, \lstinline!brusselator!, \lstinline!readiff!
  \item \lstinline!allen_cahn!, \lstinline!cahn_hilliard!
  \end{itemize}
\end{frame}

\begin{frame}
  \frametitle{Contents of model directories}
  \begin{itemize}
  \item Local integrators
    \begin{itemize}
    \item \lstinline!matrix.h!, \lstinline!residual.h!
    \end{itemize}
  \item Main programs:
    \begin{itemize}
    \item \lstinline!*.cc!
    \item Matching parameter files \lstinline!*.prm!
    \end{itemize}
  \item Consistency tests
    \begin{itemize}
    \item \lstinline!polynomial*!
    \end{itemize}
  \end{itemize}
\end{frame}

\subsection[Usage]{Suggested usage}
\frame{\tableofcontents[currentsection,subsectionstyle=show/shaded/hide]}

\begin{frame}
  \frametitle{How to use Amandus: writing}
  \begin{itemize}
  \item Solve a given equation with new domain/boundary values
    \begin{itemize}
    \item Copy main program to \lstinline!my.cc! and adjust domain
    \item Inhomogeneous boundary values:
      \begin{itemize}
      \item Always use nonlinear loop and adjust start vector
      \end{itemize}
    \end{itemize}
  \item Solve a given equation with new coefficients
    \begin{itemize}
    \item Copy integrators to new file and edit
    \item Copy main program to include new integrators
    \end{itemize}
  \end{itemize}
\end{frame}

\begin{frame}
  \frametitle{How to use Amandus: compiling}
  \begin{itemize}
  \item \lstinline!cmake! will detect your new \lstinline!*.cc! file
    \begin{itemize}
    \item Simply run in your build directory
      \begin{block}{}
        \lstinline!cmake .!
      \end{block}
    \end{itemize}
  \item Compile an application by
    \begin{block}{}
      \lstinline!make directory_executable!      
    \end{block}
  \end{itemize}
\end{frame}

\subsection{Stokes equations}
\frame{\tableofcontents[currentsection,subsectionstyle=show/shaded/hide]}

\begin{frame}
  \frametitle{The Stokes equations}
  \begin{itemize}
  \item Two quantities
    \begin{itemize}
    \item Velocity vector field $u$ from space $V$
    \item Pressure $p$ from space $Q$
    \end{itemize}
  \item Equations
    \begin{gather*}
      \arraycolsep1pt
      \begin{matrix}
        -\Delta u &-& \nabla p &=& f \\
        \nabla \!\cdot\! u &&&=&0
      \end{matrix}
    \end{gather*}
  \item Weak form: find $(u,p)$ such that
    \begin{gather*}
      (\nabla u, \nabla v) + (\nabla \!\cdot\! v,p)
      + (\nabla \!\cdot\! u,q) = (f,v)
      \quad
      \forall v\in V, q\in Q.
    \end{gather*}
  \end{itemize}
\end{frame}

\begin{frame}
  \begin{block}{Local integrator}
    \lstinputlisting[basicstyle=\footnotesize]{code/stokes-1.cc}
  \end{block}
\end{frame}

\begin{frame}
  \begin{block}{Main program}
    \lstinputlisting[basicstyle=\footnotesize]{code/stokes-2.cc}
  \end{block}
\end{frame}

\subsection{Elasticity}
\frame{\tableofcontents[currentsection,subsectionstyle=show/shaded/hide]}

\begin{frame}
  \frametitle{Elasticity}
  \begin{itemize}
  \item Lamé-Navier equations
    \begin{gather*}
      2\mu \bigl((\epsilon(u), \epsilon(v)\bigr)
      + \lambda \bigl(\nabla\!\cdot\!u,\nabla\!\cdot\!v\bigr)
      = (f,v).
    \end{gather*}
  \item Strain tensor, symmetric gradient
    \begin{gather*}
      \epsilon(u) = \frac{\nabla u + (\nabla u)^T}{2}
    \end{gather*}
  \item Inhomogeneous boundary values through Newton's method
    \begin{itemize}
    \item Solve linear system with homogeneous boundary values
    \item Have correct boundary values in start vector
    \item Newton updates will never change the boundary values
    \end{itemize}
  \end{itemize}
\end{frame}

\subsection{Eigenvalues}
\frame{\tableofcontents[currentsection,subsectionstyle=show/shaded/hide]}

\begin{frame}
  \frametitle{Eigenvalue computations}
  \begin{itemize}
  \item The variational eigenvalue problem: find $u\in V$ and $\lambda\in\mathbb C$ such that
    \begin{gather*}
      a(u,v) = \lambda (u,v)\quad\forall v\in V.
    \end{gather*}
    \begin{itemize}
    \item The bilinear form of the differential operator on the left
    \item The $L^2$-inner product on the right
    \end{itemize}
    \item Finite element formulation: find $u\in V$ and $\lambda\in\mathbb C$ such that
    \begin{gather*}
      a(u,v) = \lambda (u,v)\quad\forall v\in V.
    \end{gather*}
  \item Matrix form
    \begin{gather*}
      A u = \lambda M u
    \end{gather*}
  \end{itemize}
\end{frame}

\begin{frame}
  \begin{block}{The local integrator}
    \lstinputlisting[basicstyle=\footnotesize]{code/eigen-1.cc}
  \end{block}
\end{frame}

\begin{frame}
  \begin{block}{The main program}
    \lstinputlisting[basicstyle=\footnotesize]{code/eigen-2.cc}
  \end{block}
\end{frame}

\subsection[Allen-Kahn]{The Allen-Kahn equation}
\frame{\tableofcontents[currentsection,subsectionstyle=show/shaded/hide]}

\begin{frame}
  \frametitle{The Allen-Kahn equation}
  \begin{itemize}
  \item Nonlinear, time dependent
    \begin{gather*}
      \partial_t u -\epsilon\Delta u + u (u^2-1) = 0
    \end{gather*}
  \item The nonlinear term has two stable attractors $\pm 1$
  \item The Laplacian enforces continuity 
  \end{itemize}
\end{frame}



%%% Local Variables: 
%%% mode: latex
%%% TeX-master: "slides"
%%% End: 
