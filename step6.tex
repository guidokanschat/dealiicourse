% $Id: step6.tex 4605 2014-01-30 08:49:42Z kanschat $

%%%%%%%%%%%%%%%%%%%%%%%%%%%%%%%%%%%%%%%%%%%%%%%%%%%%%%%%%%%%%%%%%%%%%% 
%%%%%%%%%%%%%%%%%%%%%%%%%%%%%%%%%%%%%%%%%%%%%%%%%%%%%%%%%%%%%%%%%%%%%% 
\section[Adaptivity]{Adaptive refinement}
\frame{\tableofcontents[currentsection,hideothersubsections]}

%%%%%%%%%%%%%%%%%%%%%%%%%%%%%%%%%%%%%%%%%%%%%%%%%%%%%%%%%%%%%%%%%%%%%%
\begin{frame}
  \frametitle{Tutorial step 6}
  {\footnotesize{\url{http://www.dealii.org/\dealrelease/doxygen/deal.II/step_6.html}}}
  \begin{itemize}
  \item Basic error estimation
  \item Local mesh refinement strategy
  \item Hanging nodes
  \end{itemize}
\end{frame}

%%%%%%%%%%%%%%%%%%%%%%%%%%%%%%%%%%%%%%%%%%%%%%%%%%%%%%%%%%%%%%%%%%%%%%
\subsection{A basic error indicator}
\begin{frame}
  \frametitle{A basic error indicator}
  \begin{block}{\lstinline!KellyErrorEstimator<dim>::estimate(...)!}
    \begin{itemize}
    \item Error estimation based on jumps of normal derivatives
    \item Estimates roughness of the solution
    \item Assigns an ``error'' to each cell
    \end{itemize}
  \end{block}
\end{frame}

%%%%%%%%%%%%%%%%%%%%%%%%%%%%%%%%%%%%%%%%%%%%%%%%%%%%%%%%%%%%%%%%%%%%%%
\subsection{Local refinement}
\begin{frame}
  \frametitle{Local refinement}
  \begin{itemize}
  \item Use cell wise error estimates to sort cells
  \item Use a refinement strategy to mark cells
  \end{itemize}
  \begin{block}{}
    \lstinputlisting[basicstyle=\footnotesize]{tutcode/step6-1.cc}
  \end{block}
\end{frame}

%%%%%%%%%%%%%%%%%%%%%%%%%%%%%%%%%%%%%%%%%%%%%%%%%%%%%%%%%%%%%%%%%%%%%%
\subsection{Dealing with hanging nodes}
\begin{frame}
  \frametitle{Hanging nodes}
  \begin{block}{\lstinline!ConstraintMatrix!}
    \begin{itemize}
    \item FE space must be conforming
    \item Elimination of degrees from data structures is error prone
    \item Instead, eliminate on linear algebra level
    \item Constrain degrees of freedom on ``finer'' side
    \end{itemize}
  \end{block}
\end{frame}

%%%%%%%%%%%%%%%%%%%%%%%%%%%%%%%%%%%%%%%%%%%%%%%%%%%%%%%%%%%%%%%%%%%%%%
\begin{frame}
  \frametitle{Hanging nodes: additional setup}
  \begin{itemize}
  \item Setting up constraints
    \begin{block}{}
      \lstinputlisting{tutcode/step6-2.cc}
    \end{block}
  \item After creating the sparsity pattern
    \begin{block}{}
      \lstinputlisting{tutcode/step6-3.cc}
    \end{block}
  \end{itemize}
\end{frame}

%%%%%%%%%%%%%%%%%%%%%%%%%%%%%%%%%%%%%%%%%%%%%%%%%%%%%%%%%%%%%%%%%%%%%%
\begin{frame}
  \frametitle{Hanging nodes: linear algebra}
  \begin{itemize}
  \item Vectors should have zero entries at the hanging node:
    condensed form
    \begin{block}{}
      \lstinputlisting{tutcode/step6-4.cc}
    \end{block}
  \item FE functions should interpolate at the hanging node:
    distributed form
    \begin{block}{}
      \lstinputlisting{tutcode/step6-5.cc}
    \end{block}
  \end{itemize}
\end{frame}

%%%%%%%%%%%%%%%%%%%%%%%%%%%%%%%%%%%%%%%%%%%%%%%%%%%%%%%%%%%%%%%%%%%%%%
\subsection{Problems}
\begin{frame}
  \frametitle{Problems}
  \begin{enumerate}
  \item Change the mesh to \lstinline!GridGenerator::hyper_L!
  \item Use \lstinline!Functions::LSingularityFunction! for boundary
    values and zero for right hand side
  \item Compute errors after each solution using
    \begin{block}{}
      \lstinputlisting{tutcode/step5-3.cc}
    \end{block}
  \item Run the program in optimized mode using
    \begin{block}{}
      \texttt{make release}
    \end{block}
  \item What is the order of convergence? Compare to uniform refinement!
  \end{enumerate}
\end{frame}


%%% Local Variables: 
%%% mode: latex
%%% TeX-master: "slides"
%%% End: 
