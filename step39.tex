%%%%%%%%%%%%%%%%%%%%%%%%%%%%%%%%%%%%%%%%%%%%%%%%%%%%%%%%%%%%%%%%%%%%%% 
%%%%%%%%%%%%%%%%%%%%%%%%%%%%%%%%%%%%%%%%%%%%%%%%%%%%%%%%%%%%%%%%%%%%%% 
\section{DG methods}
\frame{\tableofcontents[currentsection,hideothersubsections]}

%%%%%%%%%%%%%%%%%%%%%%%%%%%%%%%%%%%%%%%%%%%%%%%%%%%%%%%%%%%%%%%%%%%%%%
\subsection{Discontinuous Galerkin methods}

\begin{frame}
  \frametitle{Origin of DG methods}
  \begin{itemize}
  \item Discontinuous Galerkin methods were developed for hyperbolic PDE
    \begin{itemize}
    \item Linear transport: Reed/Hill, LeSaint/Raviart
    \item Nonlinear conservation laws: Cockburn/Shu
    \end{itemize}
  \item They feature basis functions with no continuity across cell boundaries
  \item Soon, an extension to singularly perturbed problems became
    necessary
    \begin{gather*}
      -\varepsilon\Delta u + (w\!\cdot\!\nabla) u = f
    \end{gather*}
  \end{itemize}
\end{frame}

%%%%%%%%%%%%%%%%%%%%%%%%%%%%%%%%%%%%%%%%%%%%%%%%%%%%%%%%%%%%%%%%%%%%%%
\begin{frame}
  \frametitle{DG Methods for elliptic problems}
  \begin{itemize}
  \item Continuity is necessary for $H^1$-conformity
  \item Modify bilinear form for \textbf{consistency}
  \item Example: boundary edges
    \begin{itemize}
    \item<2-> \red Remove \black natural boundary condition
    \item<3-> \blue Symmetrize the bilinear form \black
    \item<3-> Additional terms are indefinite
    \item<4-> \green Stabilize\black
    \end{itemize}
  \end{itemize}
  \begin{align*}
    (\nabla u, \nabla v)_\Omega&
    \uncover<2->{\red -(\partial_n u, v)_{\partial \Omega}\black} 
    \uncover<3->{\blue -(u, \partial_n v)_{\partial \Omega}\black}
    \uncover<4->{\green +\tfrac{\kappa}{h}(u, v)_{\partial \Omega}\black}
    \\&=
    (f,v)_\Omega
    \uncover<3->{\blue -(u^D, \partial_n v)_{\partial \Omega}\black}
    \uncover<4->{\green +\tfrac{\kappa}{h}(u^D, v)_{\partial \Omega}\black}
  \end{align*}
\end{frame}

\begin{frame}
  \frametitle{Structure of DG code}
  \begin{itemize}
  \item Integrals over cells as before
  \item Integrals over interfaces between cells, e.g.
    \begin{gather*}
      \int_F \bigl(u^L(x)-u^R(x)\bigr) \bigl(v^L(x)-v^R(x)\bigr)
      \,ds
    \end{gather*}
    \begin{itemize}
    \item Terms generate 4 matrices!
    \end{itemize}
  \item Same integrals with one side refined (hanging nodes)
  \item Integrals over boundary faces
  \end{itemize}
\end{frame}

\begin{frame}
  \frametitle{Structure of error estimates}
  \begin{itemize}
  \item Typically two components: cells and faces
    \begin{align*}
      \eta_K &= \| h_K^2 f+\Delta u_h\|_K \\
      \eta_F &= \| h_F \operatorname{jump}(\partial_n u_h) \|_F
    \end{align*}
  \item Can be added up, such that $\eta_K$ is augmented by all
    $\eta_F$ for this cell.
  \item Results in a cell vector which can be used for refinement
    strategy
  \end{itemize}
\end{frame}

%%%%%%%%%%%%%%%%%%%%%%%%%%%%%%%%%%%%%%%%%%%%%%%%%%%%%%%%%%%%%%%%%%%%%%
%%%%%%%%%%%%%%%%%%%%%%%%%%%%%%%%%%%%%%%%%%%%%%%%%%%%%%%%%%%%%%%%%%%%%%
\subsection{Generic loops}

%%%%%%%%%%%%%%%%%%%%%%%%%%%%%%%%%%%%%%%%%%%%%%%%%%%%%%%%%%%%%%%%%%%%%%
\begin{frame}
  \frametitle{Generic loops through MeshWorker}
  \begin{itemize}
  \item Goal: see loops from an abstract point of view:
      \begin{block}{Generic linear, stationary program}
    \lstinputlisting{assembly3.pseudocode}
  \end{block}  
  \item Application program only implements the local operators
    \item No handling of degrees of freedom or target objects needed
  \end{itemize}
\end{frame}

%%%%%%%%%%%%%%%%%%%%%%%%%%%%%%%%%%%%%%%%%%%%%%%%%%%%%%%%%%%%%%%%%%%%%%
\begin{frame}
  \frametitle{Arguments to local integrator}
  \begin{itemize}
  \item\texttt{MeshWorker::DoFInfo}
    \begin{itemize}
    \item Information on the current cell and its degrees of freedom
    \item Return values in base class \texttt{LocalResults}
    \end{itemize}
  \item\texttt{MeshWorker::IntegrationInfo}
    \begin{itemize}
    \item \texttt{FEValues}
    \item Optional function values in quadrature points
    \end{itemize}
  \end{itemize}
\end{frame}

%%%%%%%%%%%%%%%%%%%%%%%%%%%%%%%%%%%%%%%%%%%%%%%%%%%%%%%%%%%%%%%%%%%%%%
\begin{frame}
  \frametitle{The local integrator class}
  \begin{block}{}\small
    \lstinputlisting{tutcode/step39-1.cc}
  \end{block}
\end{frame}

%%%%%%%%%%%%%%%%%%%%%%%%%%%%%%%%%%%%%%%%%%%%%%%%%%%%%%%%%%%%%%%%%%%%%%
\begin{frame}[t]
  \frametitle{The cell matrix}
  \begin{block}{}\footnotesize
    \lstinputlisting{code/cell_matrix.cc}
  \end{block}
\end{frame}

%%%%%%%%%%%%%%%%%%%%%%%%%%%%%%%%%%%%%%%%%%%%%%%%%%%%%%%%%%%%%%%%%%%%%%
\begin{frame}[t]
  \frametitle{The cell matrix}
  \begin{block}{}\small
    \lstinputlisting{code/cell_matrix2.cc}
  \end{block}
\end{frame}

%%%%%%%%%%%%%%%%%%%%%%%%%%%%%%%%%%%%%%%%%%%%%%%%%%%%%%%%%%%%%%%%%%%%%%
\begin{frame}
  \frametitle{The boundary matrix}
  \begin{block}{}\footnotesize
    \lstinputlisting{tutcode/step39-3.cc}
  \end{block}
\end{frame}

%%%%%%%%%%%%%%%%%%%%%%%%%%%%%%%%%%%%%%%%%%%%%%%%%%%%%%%%%%%%%%%%%%%%%%
\begin{frame}
  \frametitle{The interface matrix}
  \begin{block}{}\footnotesize
    \lstinputlisting{tutcode/step39-4.cc}
  \end{block}
\end{frame}

%%%%%%%%%%%%%%%%%%%%%%%%%%%%%%%%%%%%%%%%%%%%%%%%%%%%%%%%%%%%%%%%%%%%%%
\begin{frame}
  \frametitle{Setting up the loop I: IterationInfo}
  \begin{itemize}
  \item<1-> Handle \texttt{FEValues} objects
  \item<2-> Provide values of data vectors
  \end{itemize}
  \only<1>{\begin{block}{Assemble matrix}\small
    \lstinputlisting{tutcode/step39-5.cc}
  \end{block}}
  \only<2>{\begin{block}{Compute estimator}\footnotesize
    \lstinputlisting{tutcode/step39-5a.cc}
  \end{block}}
\end{frame}

%%%%%%%%%%%%%%%%%%%%%%%%%%%%%%%%%%%%%%%%%%%%%%%%%%%%%%%%%%%%%%%%%%%%%%
\begin{frame}
  \frametitle{Setting up the loop II: Assembler}
  \begin{block}{}\footnotesize
    \lstinputlisting{tutcode/step39-6a.cc}
  \end{block}  
  \only<1>{\begin{block}{}\footnotesize
      \lstinputlisting{tutcode/step39-6b.cc}
    \end{block}}  
  \only<2->{\begin{block}{}\footnotesize
      \lstinputlisting{tutcode/step39-6c.cc}
    \end{block}
  \begin{block}{}\footnotesize
    \lstinputlisting{tutcode/step39-6d.cc}
  \end{block}}
\end{frame}

%%%%%%%%%%%%%%%%%%%%%%%%%%%%%%%%%%%%%%%%%%%%%%%%%%%%%%%%%%%%%%%%%%%%%%
\begin{frame}
  \frametitle{Setting up the loop III: run!}
  \begin{block}{}\small
    \lstinputlisting{tutcode/step39-7.cc}
  \end{block}
  \pause
  \begin{block}{Multigrid}\small
    \lstinputlisting{tutcode/step39-7b.cc}
  \end{block}
\end{frame}

\subsection{Problems}
%%%%%%%%%%%%%%%%%%%%%%%%%%%%%%%%%%%%%%%%%%%%%%%%%%%%%%%%%%%%%%%%%%%%%%
\begin{frame}
  \frametitle{Problems}
  \begin{itemize}
  \item Add advection in the direction $(-1,-1)^T$ to the operator and
    scale the Laplacian, such that you solve
    \begin{gather*}
      -a\Delta u + \left(
      \begin{pmatrix}
        -1\\-1
      \end{pmatrix}
      \cdot \nabla\right) u = f
    \end{gather*}
    \begin{block}{Hint}
      \lstinline!namespace LocalIntegrators::Advection!      
    \end{block}
  \item Start with $a=1$ and reduce $a$.
    \begin{itemize}
    \item How does the solution change?
    \item What happens to the linear solver?
    \end{itemize}
  \end{itemize}
\end{frame}


%%% Local Variables: 
%%% mode: latex
%%% TeX-master: "slides"
%%% End: 
