% $Id: step8.tex 4605 2014-01-30 08:49:42Z kanschat $

%%%%%%%%%%%%%%%%%%%%%%%%%%%%%%%%%%%%%%%%%%%%%%%%%%%%%%%%%%%%%%%%%%%%%% 
%%%%%%%%%%%%%%%%%%%%%%%%%%%%%%%%%%%%%%%%%%%%%%%%%%%%%%%%%%%%%%%%%%%%%% 
\section[Systems]{Systems of PDE}
\frame{\tableofcontents[currentsection,hideothersubsections]}

%%%%%%%%%%%%%%%%%%%%%%%%%%%%%%%%%%%%%%%%%%%%%%%%%%%%%%%%%%%%%%%%%%%%%%
\begin{frame}
  \frametitle{Tutorial step 8}
  {\footnotesize{\url{http://www.dealii.org/\dealrelease/doxygen/deal.II/step_8.html}}}
  \begin{itemize}
  \item Lam\'e-Navier equations
  \item System elements
  \end{itemize}
\end{frame}

%%%%%%%%%%%%%%%%%%%%%%%%%%%%%%%%%%%%%%%%%%%%%%%%%%%%%%%%%%%%%%%%%%%%%%
\subsection{The Lam\'e-Navier equations}
\begin{frame}
  \frametitle{The Lam\'e-Navier equations}
  \begin{itemize}
  \item Hooke's Law for a more or less incompressible body
  \item The $\varepsilon$-tensor (symmetric gradient)
    \begin{gather*}
      \varepsilon(u) = \tfrac12
      \bigl( \nabla u + (\nabla u)^T \bigr)
    \end{gather*}
  \item Fixed-body rotation does not cause stress
  \item Divergence measures compressibility
  \item Hooke's Law
  \begin{gather*}
    -\nabla\cdot \bigl(2\mu\varepsilon(u) + \lambda \nabla\cdot
    u\bigr) = f
  \end{gather*}
  \item Weak form
    \begin{gather*}
      2\mu \int \varepsilon(u) : \varepsilon(v)\,dx
      + \lambda \int \nabla\cdot u \,\nabla\cdot v\,dx
      = \int f\cdot v\,dx
    \end{gather*}
  \end{itemize}
\end{frame}

%%%%%%%%%%%%%%%%%%%%%%%%%%%%%%%%%%%%%%%%%%%%%%%%%%%%%%%%%%%%%%%%%%%%%%
\subsection{System elements}
\begin{frame}
  \frametitle{The class FESystem}
  \begin{block}{\lstinline!FESystem!}
  \begin{itemize}
  \item Simple way of getting vector valued elements
  \item Simplifies handling of degrees of freedom
  \item Output all components at once
  \end{itemize}
  \end{block}
\end{frame}

%%%%%%%%%%%%%%%%%%%%%%%%%%%%%%%%%%%%%%%%%%%%%%%%%%%%%%%%%%%%%%%%%%%%%%
\begin{frame}
  \frametitle{Using FESystem}
  \begin{block}{}
    \lstinputlisting{tutcode/step8-0.cc}
  \end{block}
\end{frame}

%%%%%%%%%%%%%%%%%%%%%%%%%%%%%%%%%%%%%%%%%%%%%%%%%%%%%%%%%%%%%%%%%%%%%%
\begin{frame}
  \frametitle{Components of shape functions}
  \begin{itemize}
  \item ``Primitive'' systems
    \begin{itemize}
    \item The system consists only of scalar functions
    \item Each system shape function has a unique vector component
      where it may be nonzero
      \begin{block}{}
        \lstinputlisting{tutcode/step8-0a.cc}
      \end{block}
    \item All other components are zero
    \end{itemize}
  \item System contains genuine vector functions
    \begin{itemize}
    \item Shape functions may be nonzero in several components
      \begin{block}{}
        \lstinputlisting{tutcode/step8-0b.cc}
      \end{block}
    \item Requires additional loop over components
    \end{itemize}
  \end{itemize}
\end{frame}

%%%%%%%%%%%%%%%%%%%%%%%%%%%%%%%%%%%%%%%%%%%%%%%%%%%%%%%%%%%%%%%%%%%%%%
\subsection{Changes for systems}
\begin{frame}
  \frametitle{Notes on step 8}
  Compared to step 6
  \begin{itemize}
  \item \lstinline!setup_system()! is unchanged
  \item \lstinline!solve()! is unchanged
  \item \lstinline!output_results()! could have been left unchanged
  \item The ``global'' part of \lstinline!assemble_system()! is
    unchanged
    \begin{itemize}
    \item hanging nodes constraints
    \item boundary constraints
    \end{itemize}
  \item The loops in \lstinline!assemble_system()! are unchanged
  \end{itemize}
\end{frame}

%%%%%%%%%%%%%%%%%%%%%%%%%%%%%%%%%%%%%%%%%%%%%%%%%%%%%%%%%%%%%%%%%%%%%%
\begin{frame}
  \frametitle{Changes in \lstinline!assemble_system()!}
  \begin{block}{Right hand side}
    \lstinputlisting{tutcode/step8-2.cc}
  \end{block}
\end{frame}

%%%%%%%%%%%%%%%%%%%%%%%%%%%%%%%%%%%%%%%%%%%%%%%%%%%%%%%%%%%%%%%%%%%%%%
\begin{frame}
  \frametitle{Changes in \lstinline!assemble_system()!}
  \begin{block}{Matrix}
    \lstinputlisting[basicstyle=\footnotesize]{tutcode/step8-1.cc}
  \end{block}
\end{frame}

%%%%%%%%%%%%%%%%%%%%%%%%%%%%%%%%%%%%%%%%%%%%%%%%%%%%%%%%%%%%%%%%%%%%%%
\subsection{Visualization}
\begin{frame}[fragile]
  \frametitle{Visualization with gnuplot}
  \begin{block}{}
\begin{verbatim}
set style data lines
splot "solution.gpl" using ($1+$3):($2+$4)
\end{verbatim}
  \end{block}
\end{frame}

%%%%%%%%%%%%%%%%%%%%%%%%%%%%%%%%%%%%%%%%%%%%%%%%%%%%%%%%%%%%%%%%%%%%%%
\subsection{Problems}
\begin{frame}
  \frametitle{Problems}
  \begin{enumerate}
  \item Change the right hand side to zero and the boundary conditions to
    \begin{itemize}
    \item Free boundary on top and bottom
    \item Zero displacement on the left
    \item Displacement $(1,0)^T$ on the right
    \end{itemize}
    Use \lstinline!GridGenerator::hyper_rectangle! with colorizing
  \item Experiment with $\lambda$
  \end{enumerate}
\end{frame}


%%% Local Variables: 
%%% mode: latex
%%% TeX-master: "slides"
%%% End: 
