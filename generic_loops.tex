%%%%%%%%%%%%%%%%%%%%%%%%%%%%%%%%%%%%%%%%%%%%%%%%%%%%%%%%%%%%%%%%%%%%%%
%%%%%%%%%%%%%%%%%%%%%%%%%%%%%%%%%%%%%%%%%%%%%%%%%%%%%%%%%%%%%%%%%%%%%%
\subsection{Loop structures and data requirements}

%%%%%%%%%%%%%%%%%%%%%%%%%%%%%%%%%%%%%%%%%%%%%%%%%%%%%%%%%%%%%%%%%%%%%%
\begin{frame}[fragile]
  \frametitle{The assembly loop}
  \begin{gather*}
    a_{ij} = \sum_{Z\in \mathbb T_h} a(\phi_j, \phi_i)
  \end{gather*}
  \begin{block}{}
    \small
    \lstinputlisting{assembly.pseudocode}
  \end{block}  
\end{frame}

%%%%%%%%%%%%%%%%%%%%%%%%%%%%%%%%%%%%%%%%%%%%%%%%%%%%%%%%%%%%%%%%%%%%%%
\begin{frame}[fragile]
  \begin{block}{The abstract assembly loop}
    \lstinputlisting{assembly2.pseudocode}
  \end{block}
\end{frame}

%%%%%%%%%%%%%%%%%%%%%%%%%%%%%%%%%%%%%%%%%%%%%%%%%%%%%%%%%%%%%%%%%%%%%%
\begin{frame}[fragile]
  \begin{block}{The dg assembly loop}
    \lstinputlisting{assembly3.pseudocode}
  \end{block}  
\end{frame}

%%%%%%%%%%%%%%%%%%%%%%%%%%%%%%%%%%%%%%%%%%%%%%%%%%%%%%%%%%%%%%%%%%%%%%
\begin{frame}[fragile]
  \begin{block}{The vectorized dg assembly loop}
    \lstinputlisting{assembly3a.pseudocode}
  \end{block}  
\end{frame}

%%%%%%%%%%%%%%%%%%%%%%%%%%%%%%%%%%%%%%%%%%%%%%%%%%%%%%%%%%%%%%%%%%%%%%
\begin{frame}
  \frametitle{Task based parallelism by abstract loops}
  \begin{itemize}
  \item Users only provide local operators, only read local data
  \item Abstract loops enable reimplementation in the library
    \begin{itemize}
    \item Workstream with ``reliable'' accumulation, but sequential bottleneck
    \item Graph coloring and all parallel accumulation
    \item Off-loading to GPU or Xeon Phi
      \begin{itemize}
      \item Needs small code!
      \end{itemize}
    \end{itemize}
  \end{itemize}
\end{frame}


%%%%%%%%%%%%%%%%%%%%%%%%%%%%%%%%%%%%%%%%%%%%%%%%%%%%%%%%%%%%%%%%%%%%%%
\begin{frame}[fragile]
  \frametitle{FEM for linear problems}
  \begin{block}{Generic linear, stationary program}
    \lstinputlisting{linear.pseudocode}
  \end{block}
\end{frame}

%%%%%%%%%%%%%%%%%%%%%%%%%%%%%%%%%%%%%%%%%%%%%%%%%%%%%%%%%%%%%%%%%%%%%%
\begin{frame}[fragile]
  \frametitle{FEM for nonlinear problems}
  \begin{block}{Generic nonlinear, stationary program}
    \lstinputlisting{nonlinear.pseudocode}
  \end{block}
\end{frame}

%%%%%%%%%%%%%%%%%%%%%%%%%%%%%%%%%%%%%%%%%%%%%%%%%%%%%%%%%%%%%%%%%%%%%%
\begin{frame}[fragile]
  \frametitle{FEM for instationary nonlinear problems}
  \begin{block}{Generic nonlinear, instationary program}
    \lstinputlisting{timestepping.pseudocode}
  \end{block}
\end{frame}

\begin{frame}
  \frametitle{Generic loop interface in deal.II}
  \lstinline!MeshWorker! class uses three objects
  \begin{enumerate}
  \item Local operator
    \begin{itemize}
    \item  provided by user
    \end{itemize}
  \item Output object
      \begin{itemize}
      \item Matrix data
      \item Vector data
      \item Functionals and estimators
      \end{itemize}
  \item Data object
      \begin{itemize}
      \item Receives any number and any kind of (named) data
      \item Automatically evaluates finite element functions on each cell
      \end{itemize}
  \end{enumerate}
\end{frame}

\subsection{Local operators}

%%%%%%%%%%%%%%%%%%%%%%%%%%%%%%%%%%%%%%%%%%%%%%%%%%%%%%%%%%%%%%%%%%%%%%
\begin{frame}
  \frametitle{Local operators (class style)}
  \begin{block}{}
    \lstinputlisting[basicstyle=\footnotesize]{code/local_integrator.h}
  \end{block}
\end{frame}

%%%%%%%%%%%%%%%%%%%%%%%%%%%%%%%%%%%%%%%%%%%%%%%%%%%%%%%%%%%%%%%%%%%%%%
\begin{frame}
  \frametitle{Examples for local operators}
  \begin{block}{Laplace matrix}
    \lstinputlisting[basicstyle=\footnotesize]{code/laplace_matrix.h}
  \end{block}
\end{frame}

%%%%%%%%%%%%%%%%%%%%%%%%%%%%%%%%%%%%%%%%%%%%%%%%%%%%%%%%%%%%%%%%%%%%%%
\begin{frame}
  \frametitle{Examples for local operators}
  \begin{block}{The lambda version}
    \lstinputlisting[basicstyle=\footnotesize]{code/laplace_lambda.h}
  \end{block}
\end{frame}

%%%%%%%%%%%%%%%%%%%%%%%%%%%%%%%%%%%%%%%%%%%%%%%%%%%%%%%%%%%%%%%%%%%%%%
\begin{frame}
  \frametitle{Examples for local operators}
  \begin{block}{Allen-Cahn Jacobian}
    \lstinputlisting[basicstyle=\footnotesize]{code/allencahn_matrix.h}
  \end{block}
\end{frame}

%%%%%%%%%%%%%%%%%%%%%%%%%%%%%%%%%%%%%%%%%%%%%%%%%%%%%%%%%%%%%%%%%%%%%%
\begin{frame}
  \frametitle{Examples for local operators}
  \begin{block}{Allen-Cahn residual}
    \lstinputlisting[basicstyle=\footnotesize]{code/allencahn_residual.h}    
  \end{block}
\end{frame}

%%%%%%%%%%%%%%%%%%%%%%%%%%%%%%%%%%%%%%%%%%%%%%%%%%%%%%%%%%%%%%%%%%%%%%
\begin{frame}[t]
  \frametitle{Assembling the matrix}
  \lstinputlisting[basicstyle=\footnotesize]{code/matrix.cc}
\end{frame}

%%%%%%%%%%%%%%%%%%%%%%%%%%%%%%%%%%%%%%%%%%%%%%%%%%%%%%%%%%%%%%%%%%%%%%
\begin{frame}[t]
  \frametitle{Assembling the multilevel matrices}
  \lstinputlisting[basicstyle=\footnotesize]{code/mgmatrix.cc}
\end{frame}

%%%%%%%%%%%%%%%%%%%%%%%%%%%%%%%%%%%%%%%%%%%%%%%%%%%%%%%%%%%%%%%%%%%%%%
\begin{frame}[t]
  \frametitle{Assembling the right hand side}
  \lstinputlisting[basicstyle=\footnotesize]{code/rhs.cc}
\end{frame}

