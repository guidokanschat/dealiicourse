%%%%%%%%%%%%%%%%%%%%%%%%%%%%%%%%%%%%%%%%%%%%%%%%%%%%%%%%%%%%%%%%%%%%%% 
%%%%%%%%%%%%%%%%%%%%%%%%%%%%%%%%%%%%%%%%%%%%%%%%%%%%%%%%%%%%%%%%%%%%%% 
\section[Poisson]{Solving a Poisson problem}
\frame{\tableofcontents[currentsection,hideothersubsections]}

%%%%%%%%%%%%%%%%%%%%%%%%%%%%%%%%%%%%%%%%%%%%%%%%%%%%%%%%%%%%%%%%%%%%%%
\begin{frame}
  \frametitle{Tutorial Step 3}
  {\footnotesize{\url{http://www.dealii.org/\dealrelease/doxygen/deal.II/step_3.html}}}
  \begin{itemize}
  \item Defining a class for the application
  \item Creating a finite element space on a mesh
  \item Building a system matrix and right hand side
  \item Solving the discrete linear system
  \item Visualization
  \end{itemize}
\end{frame}

%%%%%%%%%%%%%%%%%%%%%%%%%%%%%%%%%%%%%%%%%%%%%%%%%%%%%%%%%%%%%%%%%%%%%%
\subsection{Application classes}
\begin{frame}
  \begin{block}{The application class}
    \lstinputlisting{tutcode/step3-1.cc}
  \end{block}
\end{frame}

%%%%%%%%%%%%%%%%%%%%%%%%%%%%%%%%%%%%%%%%%%%%%%%%%%%%%%%%%%%%%%%%%%%%%%
\begin{frame}
  \begin{block}{The private parts of the application class}
    \lstinputlisting[basicstyle=\footnotesize]{tutcode/step3-2.cc}    
  \end{block}
\end{frame}

%%%%%%%%%%%%%%%%%%%%%%%%%%%%%%%%%%%%%%%%%%%%%%%%%%%%%%%%%%%%%%%%%%%%%%
\begin{frame}
  \begin{block}{The constructor}
  \lstinputlisting[basicstyle=\small]{tutcode/step3-3.cc}
  \end{block}
\end{frame}

%%%%%%%%%%%%%%%%%%%%%%%%%%%%%%%%%%%%%%%%%%%%%%%%%%%%%%%%%%%%%%%%%%%%%%
\begin{frame}
  \frametitle{run!}
  \begin{block}{}
  \lstinputlisting[basicstyle=\small]{tutcode/step3-5.cc}
  \end{block}
\end{frame}

%%%%%%%%%%%%%%%%%%%%%%%%%%%%%%%%%%%%%%%%%%%%%%%%%%%%%%%%%%%%%%%%%%%%%%
\subsection{Finite element spaces}
\begin{frame}
  \begin{block}{Creating the mesh}
  \lstinputlisting[basicstyle=\small]{tutcode/step3-4a.cc}
  \end{block}
\end{frame}

%%%%%%%%%%%%%%%%%%%%%%%%%%%%%%%%%%%%%%%%%%%%%%%%%%%%%%%%%%%%%%%%%%%%%%
\begin{frame}
  \begin{block}{Generate the finite element space}
  \lstinputlisting[basicstyle=\small]{tutcode/step3-4.cc}
  \end{block}
  \begin{itemize}
  \item Mapping between linear algebra and functions
  \end{itemize}
\end{frame}

%%%%%%%%%%%%%%%%%%%%%%%%%%%%%%%%%%%%%%%%%%%%%%%%%%%%%%%%%%%%%%%%%%%%%%
\subsection{The linear system}
\begin{frame}
  \frametitle{Assembling the system}
  \begin{itemize}
  \item<+-> A loop over all active cells
  \item<+-> Creating a cell matrix and cell vector
  \item<+-> Access to shape function values in quadrature points:
    \begin{block}{\lstinline!FEValues: public FEValuesBase!}
    \begin{itemize}
    \item function values and derivatives
    \item mapped geometry
    \item Allows for optimization when computing values
    \end{itemize}
    \end{block}
  \end{itemize}
\end{frame}

%%%%%%%%%%%%%%%%%%%%%%%%%%%%%%%%%%%%%%%%%%%%%%%%%%%%%%%%%%%%%%%%%%%%%%
\begin{frame}
  \begin{block}{Using FEValues}
  \lstinputlisting[basicstyle=\footnotesize]{tutcode/step3-6.cc}
  \end{block}
\end{frame}

%%%%%%%%%%%%%%%%%%%%%%%%%%%%%%%%%%%%%%%%%%%%%%%%%%%%%%%%%%%%%%%%%%%%%%
\begin{frame}
  \begin{block}{Integrating the matrix}
  \lstinputlisting[basicstyle=\footnotesize]{tutcode/step3-6a.cc}
  \end{block}
\end{frame}

%%%%%%%%%%%%%%%%%%%%%%%%%%%%%%%%%%%%%%%%%%%%%%%%%%%%%%%%%%%%%%%%%%%%%%
\begin{frame}
  \begin{block}{Boundary conditions}
  \lstinputlisting[basicstyle=\small]{tutcode/step3-7.cc}
  \end{block}
\end{frame}

%%%%%%%%%%%%%%%%%%%%%%%%%%%%%%%%%%%%%%%%%%%%%%%%%%%%%%%%%%%%%%%%%%%%%%
\subsection{Solving}
\begin{frame}
  \begin{block}{Solving the linear system}
  \lstinputlisting[basicstyle=\small]{tutcode/step3-8.cc}
  \end{block}
\end{frame}

%%%%%%%%%%%%%%%%%%%%%%%%%%%%%%%%%%%%%%%%%%%%%%%%%%%%%%%%%%%%%%%%%%%%%%
\subsection{Visualization}
\begin{frame}
  \frametitle{Visualization}
  \begin{block}{}
    \lstinputlisting[basicstyle=\small]{tutcode/step3-9.cc}
  \end{block}
  \begin{itemize}
  \item Supports gnuplot, VTK, UCD, OpenDX, TecPlot, Povray, VTU, EPS
  \end{itemize}
\end{frame}

%%%%%%%%%%%%%%%%%%%%%%%%%%%%%%%%%%%%%%%%%%%%%%%%%%%%%%%%%%%%%%%%%%%%%%
\subsection{Problems}
\begin{frame}
  \frametitle{Step 3 Problems: Poisson equation}
  \begin{enumerate}
  \item Solve Helmholtz equation
    \begin{gather*}
      \int \nabla u\cdot\nabla v\,dx - \kappa^2 \int uv\,dx= \int 1v\,dx
    \end{gather*}
    with $\kappa = 1, 4, 10$.
  \item Change the linear solver if necessary
  \item What happens if you refine more?
  \item Change the boundary condition to the natural boundary condition
    \begin{gather*}
      \partial_n u = 0
    \end{gather*}
    on part of the boundary.
  \end{enumerate}
\end{frame}


%%% Local Variables: 
%%% mode: latex
%%% TeX-master: "slides"
%%% End: 
