%%%%%%%%%%%%%%%%%%%%%%%%%%%%%%%%%%%%%%%%%%%%%%%%%%%%%%%%%%%%%%%%%%%%%%
%%%%%%%%%%%%%%%%%%%%%%%%%%%%%%%%%%%%%%%%%%%%%%%%%%%%%%%%%%%%%%%%%%%%%%
\section[Virtual Box]{Setting up the virtual box}
\frame{\tableofcontents[currentsection,hideothersubsections]}

\subsection{Installing the virtual machine}
\begin{frame}
  \frametitle{Downloading the virtual machine}
  \begin{enumerate}
  \item Install the Oracle VM Virtual Box (Open Source)
    \begin{itemize}
    \item \url{www.virtualbox.org}
    \item Packaged with many Linux distributions
    \end{itemize}
    \item Download the Virtual Machine Image from \url{dealii.org/download.html}
    \item Start the Virtual Box manager and import the \texttt{.ova} file
      \begin{itemize}
      \item Adjust the settings for CPUs, Memory to be less than what
        your machine has
      \item Delete the shared folder entry
      \end{itemize}
    \item Start the newly imported virtual image
  \end{enumerate}
\end{frame}

\subsection{Preparing dealii for our course}
\begin{frame}
  \frametitle{Preparing dealii for our course}
  \begin{enumerate}
  \item Start the terminal (second menu item in the top bar)
  \item Run the following
    \begin{block}{}
      \lstinputlisting{code/vmsetup.sh}      
    \end{block}
  \item Edit \lstinline!setup.sh! and remove the line that contains
    \begin{block}{}
      \lstinline!git checkout v8.4.1!
    \end{block}
    \item Run \lstinline!setup.sh! 
  \item Go and have coffee, lunch, dinner, etc.
  \end{enumerate}
\end{frame}

\subsection{Downloading and setting up amandus}
\begin{frame}
  \frametitle{Downloading and setting up amandus}
  \begin{enumerate}
  \item From your home directory, run the following:
    \begin{block}{}
      \lstinputlisting[basicstyle=\footnotesize]{code/vmamandus.sh}
    \end{block}
  \end{enumerate}
\end{frame}

%%%%%%%%%%%%%%%%%%%%%%%%%%%%%%%%%%%%%%%%%%%%%%%%%%%%%%%%%%%%%%%%%%%%%%
%%%%%%%%%%%%%%%%%%%%%%%%%%%%%%%%%%%%%%%%%%%%%%%%%%%%%%%%%%%%%%%%%%%%%%
\section[Installing]{Installing deal.II without virtual box}
\frame{\tableofcontents[currentsection,hideothersubsections]}
\subsection{Requirements}

%%%%%%%%%%%%%%%%%%%%%%%%%%%%%%%%%%%%%%%%%%%%%%%%%%%%%%%%%%%%%%%%%%%%%%
\begin{frame}
  \frametitle{Operating system}
  \begin{itemize}
  \item Linux (e.g. Ubuntu 14.04 or higher)
    \begin{itemize}
    \item g++, 
    \end{itemize}
  \item Mac OS X
    \begin{itemize}
    \item Install current XCode or MAC Ports
    \end{itemize}
  \item Windows with Virtual Box
  \end{itemize}
  These will 
\end{frame}

%%%%%%%%%%%%%%%%%%%%%%%%%%%%%%%%%%%%%%%%%%%%%%%%%%%%%%%%%%%%%%%%%%%%%%
\begin{frame}
  \frametitle{Additional software}
  \begin{itemize}
  \item CMake (3.0 or higher)
  \item perl (5.x or higher)
  \item make or ninja as required by CMake
  \item Some visualization software
    \item Optional: 
    \begin{itemize}
    \item LAPACK, Arpack
    \item doxygen and graphviz (for documentation only)
    \item MPI, p4est (for distributed computing)
    \item Trilinos, Petsc, Slepc (parallel linear algebra)
    \end{itemize}
  \end{itemize}
\end{frame}

%%%%%%%%%%%%%%%%%%%%%%%%%%%%%%%%%%%%%%%%%%%%%%%%%%%%%%%%%%%%%%%%%%%%%%
%%%%%%%%%%%%%%%%%%%%%%%%%%%%%%%%%%%%%%%%%%%%%%%%%%%%%%%%%%%%%%%%%%%%%%
\subsection{Downloading}

%%%%%%%%%%%%%%%%%%%%%%%%%%%%%%%%%%%%%%%%%%%%%%%%%%%%%%%%%%%%%%%%%%%%%%
\begin{frame}
  \frametitle{Downloading deal.II releases}
  \begin{itemize}
  \item Go to \texttt{\myurl{http://www.dealii.org}{www.dealii.org}}
  \item In the navigation menu to the left, follow
    \myurl{http://www.dealii.org/download/index.html}{Download}
  \item Choose the release number (usually the latest)
  \item Choose the package
    \begin{itemize}
    \item Full library
    \item Premade documentation
    \end{itemize}
  \item Download the file (in \texttt{.tar.gz} format)
  \item In an appropriate place, unpack and untar the file, which will
    create a subdirectory \texttt{deal.II}.
  \end{itemize}
\end{frame}

%%%%%%%%%%%%%%%%%%%%%%%%%%%%%%%%%%%%%%%%%%%%%%%%%%%%%%%%%%%%%%%%%%%%%%
\begin{frame}
  \frametitle{Access the developing archive}
  \begin{itemize}
  \item The current state of deal.II is on github at\\
    \texttt{https://github.com/dealii/dealii.git}
  \item First time call\\
    \texttt{\footnotesize git clone https://github.com/dealii/dealii.git}
  \item make sure you regularly call\\
    \texttt{git pull}
  \item Release or subversion archive?
    \begin{itemize}
    \item Releases are tested!
    \item Archive has the most current features
    \end{itemize}
    \pause
    \item Contributing to deal.II
      \begin{itemize}
      \item Create an account and a fork on github
      \item Familiarize yourself with the git workflow
      \end{itemize}
  \end{itemize}
\end{frame}

%%%%%%%%%%%%%%%%%%%%%%%%%%%%%%%%%%%%%%%%%%%%%%%%%%%%%%%%%%%%%%%%%%%%%%
%%%%%%%%%%%%%%%%%%%%%%%%%%%%%%%%%%%%%%%%%%%%%%%%%%%%%%%%%%%%%%%%%%%%%%
\subsection{Configuring}

%%%%%%%%%%%%%%%%%%%%%%%%%%%%%%%%%%%%%%%%%%%%%%%%%%%%%%%%%%%%%%%%%%%%%%
\begin{frame}[fragile]
  \frametitle{Configuring and installing}
  \begin{itemize}
  \item Unpack library into directory of your choice
  \item Typically:
\begin{lstlisting}[language=bash,basicstyle=\ttfamily,keywordstyle=\ttfamily]
mkdir dealii
cd dealii
tar xzf /download/path/deal.II-%\dealrelease%.tar.gz
mkdir build
cd build
cmake -DCMAKE_INSTALL_PREFIX=../installed \
    ../deal.II
make install -j 4
\end{lstlisting}
\item Additional configuration options
    \begin{itemize}
    \item Disable/enable features
    \item Add interfaces to external libraries
    \item Menu with \texttt{ccmake .}
    \end{itemize}
  \end{itemize}
\end{frame}

%%%%%%%%%%%%%%%%%%%%%%%%%%%%%%%%%%%%%%%%%%%%%%%%%%%%%%%%%%%%%%%%%%%%%%
%%%%%%%%%%%%%%%%%%%%%%%%%%%%%%%%%%%%%%%%%%%%%%%%%%%%%%%%%%%%%%%%%%%%%%
\subsection{Running your first program}

%%%%%%%%%%%%%%%%%%%%%%%%%%%%%%%%%%%%%%%%%%%%%%%%%%%%%%%%%%%%%%%%%%%%%%
\begin{frame}[fragile]
  \frametitle{Running the examples}
\begin{lstlisting}[language=bash,basicstyle=\ttfamily,keywordstyle=\ttfamily]
cd dealii/installed/examples/step-1
cmake .
make run
evince grid-2.eps
\end{lstlisting}
\end{frame}


%%% Local Variables: 
%%% mode: latex
%%% TeX-master: "slides"
%%% End: 
