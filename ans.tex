\section[Why Software?]{Why should mathematicians use high level software?}
\frame{\tableofcontents[currentsection,hideothersubsections]}

%%%%%%%%%%%%%%%%%%%%%%%%%%%%%%%%%%%%%%%%%%%%%%%%%%%%%%%%%%%%%%%%%%%%%%
\begin{frame}
  \frametitle{State of Scientific Computation}
  % [Apologies for the fact that this is mostly from a numerical PDE
  % perspective]
  % \vfill
  \begin{itemize}
  \item Numerical methods become more complex
    \begin{itemize}
    \item linear/nonlinear solvers
    \item multigrid
    \item adaptivity
    \item parameter estimation/optimization
    \end{itemize}
  \item We have the methods to tackle challenging applications
  \item Modern computers add challenges when it comes to ``efficient''
    implementation
  \end{itemize}
  \pause
  \begin{block}{}
``The future of Scientific Computing has just started'' (P.~Deuflhard, 2007)    
  \end{block}

\end{frame}

\begin{frame}
  \frametitle{Academic Software Development}
  The ``graduate student cycle''
  \begin{itemize}
  \item write a finite element program as a class project
  \item Extend it to provide numerical experiments for PhD project
  \item Publish results
  \item Successor begins from scratch
  \end{itemize}
  \pause
  \begin{block}{}
    Scientific computation remains limited to project which can be
    solved during the ``life time'' of a PhD student.
  \end{block}
  \pause
  Exceptions
  \begin{itemize}
  \item National labs
  \item Few centers of scientific computation
  \end{itemize}
\end{frame}

\begin{frame}
  \frametitle{Extension of the Graduate Student Cycle}
  \begin{itemize}
  \item Student becomes professor
  \item and makes own students continue developing the code
  \item New functionality added over generations, resulting in
    \begin{itemize}
    \item  Monolithic software of 100,000s of statements,
    \item  solving multiphysics applications
      \begin{itemize}
      \item clashes of different coding styles
      \item missing or insufficient documentation
      \item lack of structure
      \item reliability is hard to maintain
      \end{itemize}
    \end{itemize}
  \item Simulation becomes black magic
  \end{itemize}
\end{frame}

\begin{frame}
  \frametitle{Breaking the Graduate Student Cycle}
  \begin{enumerate}
  \item Highly integrated libraries
    \begin{itemize}
    \item Programs start on a higher level
    \item PhD students can solve more complex problems
    \item programs get shorter, so they can be forwarded and edited
    \item library functionality is documented and tested regularly
    \end{itemize}
  \item Publication of application software
    \begin{itemize}
    \item Recognition of intellectual achievement in good software
    \item Readers can run programs and verify results
    \item Software builds on previous publications, much like theorems do
    \end{itemize}
  \end{enumerate}
\end{frame}

%% \begin{frame}
%%   \frametitle{Archive of Numerical Software}
%%   \begin{center}
%%     \href{http://journals.tdl.org/ans}{\texttt{\large http://journals.tdl.org/ans}}
%%   \end{center}
%%   \begin{itemize}
%%   \item Publication of well-documented, well-structured software
%%   \item ``Short'' programs based on established libraries
%%   \item Reported results must be reproducible
%%   \item Community based, no additional cost to you library
%%   \end{itemize}
%% \end{frame}

%% \begin{frame}
%%   \frametitle{Mission Statement}

%%   The Archive of Numerical Software aims to provide a venue to promote
%%   the design, creation, use and re-use of high level applications and
%%   high quality software packages for the implementation of numerical
%%   methods.
%% \end{frame}


%% \begin{frame}
%%   \frametitle{Reproducibility}

%%   All results presented in the article must be reproducible using only
%%   a few, simple commands and basic editing. We encourage to provide
%%   scripts which automatically set parameters and execute examples, but
%%   a confined section in the source code where parameters can be easily
%%   edited is acceptable.

%%   In furthering the journal's goal of promoting the use and re-use of
%%   code, codes should be compatible with widely used and widely
%%   available software (operating systems, compilers) and hardware
%%   (CPUs).
%% \end{frame}

%%% Local Variables: 
%%% mode: latex
%%% TeX-master: "slides"
%%% End: 
